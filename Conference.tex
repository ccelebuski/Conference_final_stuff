%%%%%%%%%%%%%%%%%%%%%%%%%%%%%%%%%%%%%%%%%
% Short Sectioned Assignment
% LaTeX Template
% Version 1.0 (5/5/12)
%
% This template has been downloaded from:
% http://www.LaTeXTemplates.com
%
% Original author:
% Frits Wenneker (http://www.howtotex.com)
%
% License:
% CC BY-NC-SA 3.0 (http://creativecommons.org/licenses/by-nc-sa/3.0/)
%
%%%%%%%%%%%%%%%%%%%%%%%%%%%%%%%%%%%%%%%%%

%----------------------------------------------------------------------------------------
%	PACKAGES AND OTHER DOCUMENT CONFIGURATIONS
%----------------------------------------------------------------------------------------

\documentclass[paper=a4, fontsize=11pt]{scrartcl} % A4 paper and 11pt font size

\usepackage[T1]{fontenc} % Use 8-bit encoding that has 256 glyphs
\usepackage{fourier} % Use the Adobe Utopia font for the document - comment this line to return to the LaTeX default
\usepackage[english]{babel} % English language/hyphenation
\usepackage{amsmath,amsfonts,amsthm} % Math packages
\usepackage{mathtools}

\usepackage{lipsum} 
\usepackage{framed}% Used for inserting dummy 'Lorem ipsum' text into the template

\usepackage{sectsty} % Allows customizing section commands
\allsectionsfont{\centering \normalfont\scshape} % Make all sections centered, the default font and small caps

\usepackage{fancyhdr} % Custom headers and footers
\pagestyle{fancyplain} % Makes all pages in the document conform to the custom headers and footers
\fancyhead{} % No page header - if you want one, create it in the same way as the footers below
\fancyfoot[L]{} % Empty left footer
\fancyfoot[C]{} % Empty center footer
\fancyfoot[R]{\thepage} % Page numbering for right footer
\renewcommand{\headrulewidth}{0pt} % Remove header underlines
\renewcommand{\footrulewidth}{0pt} % Remove footer underlines
\setlength{\headheight}{13.6pt} % Customize the height of the header

\numberwithin{equation}{section} % Number equations within sections (i.e. 1.1, 1.2, 2.1, 2.2 instead of 1, 2, 3, 4)
\numberwithin{figure}{section} % Number figures within sections (i.e. 1.1, 1.2, 2.1, 2.2 instead of 1, 2, 3, 4)
\numberwithin{table}{section} % Number tables within sections (i.e. 1.1, 1.2, 2.1, 2.2 instead of 1, 2, 3, 4)

\setlength\parindent{0pt} % Removes all indentation from paragraphs - comment this line for an assignment with lots of text

%----------------------------------------------------------------------------------------
%	TITLE SECTION
%----------------------------------------------------------------------------------------

\newcommand{\horrule}[1]{\rule{\linewidth}{#1}} % Create horizontal rule command with 1 argument of height

\title{	
\normalfont \normalsize 
\textsc{Sarah Lawrence College} \\ [25pt] % Your university, school and/or department name(s)
\horrule{0.5pt} \\[0.4cm] % Thin top horizontal rule
\huge Orbits In Processing \\ % The assignment title
\horrule{2pt} \\[0.5cm] % Thick bottom horizontal rule
}

\author{Calvin Celebuski} % Your name

\date{\normalsize\today} % Today's date or a custom date

\begin{document}

\maketitle % Print the title

%----------------------------------------------------------------------------------------
%	PROBLEM 1
%----------------------------------------------------------------------------------------

\begin{framed}
\footnotesize
\begin{verbatim}
float x,y,vx,vy,a,r,ax,ay,x2,y2,v,KE,PE,ME;
float dx,dy,dr,dth,drh,drhx,drhy,rhx,rhy,xp,yp,A;
\end{verbatim}
\end{framed}
\normalsize
\begin{verbatim}
Here, I establish the variables.
\end{verbatim}
\begin{framed}
\footnotesize
\begin{verbatim}
void setup(){
  frameRate(600);
  x=0;
  y=100;
  x2=width/2;
  y2=height/2;
  vx=1.1;
  vy=-0.01;
  size(1280,730);
}
\end{verbatim}
\end{framed}
\normalsize
\begin{verbatim}
from x=0 to vy=-0.01 is where I set the initial conditions
\end{verbatim}
\begin{framed}
\footnotesize
\begin{verbatim}
void draw(){
  background(0);
  stroke(0,0,50);
  fill(0,0,255);
  ellipse(x2,y2,20,20);
\end{verbatim}
\end{framed}
\normalsize
\begin{verbatim}
That's the blue body.  It stays put at the center of the simulation.

Below, I include some things for user input.  Allowing the user to have
an effect on the red body's movement and to change the frame rate.
\end{verbatim}
\begin{framed}
\footnotesize
\begin{verbatim}
  stroke(0,0,255);
  if (keyPressed){
    if(key=='w'){ //forward boost
      vx=vx+A*drhx;
      vy=vy+A*drhy;
      ellipse(x+x2-(3*drhx),y+y2-(3*drhy),20,20);
    }
    if(key=='a'){ //left boost
      vx=vx+A*drhy;
      vy=vy-A*drhx;
      ellipse(x+x2-(3*drhy),y+y2+(3*drhx),20,20);
    }
    if(key=='s'){ //backward boost
      if(v>0.01){
        vx=vx-A*drhx;
        vy=vy-A*drhy;
        ellipse(x+x2+(3*drhx),y+y2+(3*drhy),20,20);
      }
      if(v<=0.01){
        vx=0;
        vy=0;
        ax=0;
        ay=0;
        ellipse(x+x2,y+y2,24,24);
      }
    }
    if(key=='d'){ //right boost
      vx=vx-A*drhy;
      vy=vy+A*drhx;
      ellipse(x+x2+(3*drhy),y+y2-(3*drhx),20,20);
    }
    if(key=='z'){
      frameRate(1);
    }
    if(key=='x'){
      frameRate(60);
    }
    if(key=='c'){
      frameRate(600);
    }
  }
  stroke(50,0,0);
  fill(255,0,0);
  ellipse(x+x2,y+y2,20,20);
\end{verbatim}
\end{framed}
\normalsize
\begin{verbatim}
Here I establish the red body, setting it relative to the blue body by
making its position x+x2, y+y2.
\end{verbatim}
\begin{framed}
\footnotesize
\begin{verbatim}
  r=sqrt((x*x)+(y*y));
  A=.002; //acceleration caused by user input
  a=-100/(r*r); //acceleration
  ax=a*x/r; //acceleration in the x direction
  ay=a*y/r; //acceleration in the y direction
  vx=vx+ax; //update velocity in the x direction
  vy=vy+ay; //update velocity in the y direction
  x=x+vx; //update x position
  y=y+vy; //update y position
  xp=x-vx; //previous x position
  yp=y-vy; //previous y position
  dx=x-xp; //delta x
  dy=y-yp; //delta y
  v=sqrt(vx*vx+vy*vy); //velocity
  dr=sqrt(dx*dx+dy*dy); //delta r
  drhx=dx/dr; //delta rhat x = cosTheta
  drhy=dy/dr; //delta rhat y = sinTheta
\end{verbatim}
\end{framed}
\normalsize
\begin{verbatim}
Below I determine the Kinetic Energy, Potential Energy, and Total
Mechanical Energy and display them and some other numbers on screen.
\end{verbatim}
\begin{framed}
\footnotesize
\begin{verbatim}
  KE=v*v/2.;
  PE=-100/(r);
  ME=KE+PE;
  fill(255);
  text("velocity="+str(v),40,40);
  text("acceleration="+str(a*-1),40,50);
  text("distance="+str(r),40,60);
  text("Kinetic Energy="+str(KE),40,70);
  text("Potential Energy="+str(PE),40,80);
  text("Mechanical Energy="+str(ME),40,90);
  //if(x+x2>=x2-10&&x+x2<=x2+10&&y+y2<=y2+21&&y+y2>=y2){
  //  vy=vy*-1;
  //}
}
\end{verbatim}
\end{framed}
\normalsize
That last part was a failed attempt to get the red body to bounce off of the blue body.  I set
it inactive with the two slashes.

\end{document}